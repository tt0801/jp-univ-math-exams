\documentclass[dvipdfmx,a4paper]{jsarticle}
\pagestyle{plain}
\usepackage{amsmath}
\usepackage{amsthm}
\usepackage{amssymb}
\usepackage{graphicx}
\usepackage{tikz}
\usepackage{tkz-euclide}
\usepackage{here}
\usepackage{cancel}

\usetikzlibrary{angles}
\usetikzlibrary{calc}

\newcommand{\R}{\mathbb{R}}
\newcommand{\C}{\mathbb{C}}
\newcommand{\Z}{\mathbb{Z}}
\newcommand{\N}{\mathbb{N}}
\newcommand{\Q}{\mathbb{Q}}
\newcommand{\Lra}{\Leftrightarrow}
\newcommand{\al}{\alpha}
\newcommand{\be}{\beta}
\newcommand{\ga}{\gamma}
\newcommand{\om}{\omega}
\newcommand{\De}{\Delta}
\newcommand{\oraw}{\overrightarrow}
\newcommand{\posv}[1]{\overrightarrow{\mathrm{#1}}}
\newcommand{\comb}[2]{{}_{#1}\mathrm{C}_{#2}}
\newcommand{\perm}[2]{{}_{#1}\mathrm{P}_{#2}}
\newcommand{\bs}{\backslash}
\newcommand{\2}{I\hspace{-1pt}I}
\newcommand{\3}{I\hspace{-1pt}I\hspace{-1pt}I}

\newtheorem{cs}{Case}
\newtheorem{cas}{Case}
\newtheorem{case}{Case}
\newtheorem{apf}{別解}
\newtheorem{anpf}{別解}

% maketitle
\title{2023年度京大数学(文系)の問題}
\author{tt0801}
\date{\today}    

\begin{document}
    \maketitle
    \section{大問1}
    次の各問に答えよ. 

    \begin{itemize}
        \item [問1] \quad $n$ を自然数とする. 1個のさいころを $n$ 回投げるとき, 出た目の積が 5 で割り切れる確率を求めよ. 
        \item [問2] \quad 次の式の分母を有理化し, 分母に 3 乗根の記号が含まれない式として表せ. 
            \[
                \frac{55}{2 \sqrt[3]{9} + \sqrt[3]{3} + 5}
            \]
    \end{itemize}

    
    \section{大問2}
    空間内の4点 O, A, B, Cは同一平面上にないとする. 点D, P, Qを次のように定める. 
    点Dは$\posv{OD} = \posv{OA} + 2\posv{OB} + 3\posv{OC}$ を満たし,   
    点Pは線分OAを$1 : 2$に内分し, 点Qは線分OBの中点である. さらに,   
    直線OD上の点Rを, 直線QRと直線PCが交点を持つように定める.   

    このとき, 線分ORの長さと線分RDの長さの比$\mathrm{OR} : \mathrm{RD}$を求めよ. 

    \section{大問3}
    \begin{itemize}
        \item [(1)] $\cos 2\theta$と$\cos 3\theta$を$\cos \theta$の式として表せ. 
        \item [(2)] 半径1の円に内接する正五角形の一辺の長さが1.15より大きいか否かを理由を付けて判定せよ. 
    \end{itemize}


    \section{大問4}
    数列 $\{a_n\}$ は次の条件を満たしている. 

    \[
    a_1 = 3, \quad a_n = \frac{S_n}{n} + (n-1) \cdot 2^n \quad (n = 2, 3, 4, \ldots)
    \]

    ただし, $S_n = a_1 + a_2 + \cdots + a_n$ である. このとき, 数列 $\{a_n\}$ の一般項を求めよ. 

    \section{大問5}
    整式 $f(x)$ が恒等式
    \begin{equation*}
        f(x) + \int_{-1}^1 (x-y)^2 f(y) dy = 2x^2 + x + \frac{5}{3}
    \end{equation*}
    を満たすとき, $f(x)$ を求めよ. 


\end{document}