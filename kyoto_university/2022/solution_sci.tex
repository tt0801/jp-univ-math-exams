\documentclass[dvipdfmx,a4paper]{jsarticle}
    \pagestyle{plain}
    \usepackage{amsmath}
    \usepackage{amsthm}
    \usepackage{amssymb}
    \usepackage{graphicx}
    \usepackage{tikz}
    \usepackage{tkz-euclide}
    \usepackage{here}
    \usepackage{cancel}

    \usetikzlibrary{angles}

    \newcommand{\R}{\mathbb{R}}
    \newcommand{\C}{\mathbb{C}}
    \newcommand{\Z}{\mathbb{Z}}
    \newcommand{\N}{\mathbb{N}}
    \newcommand{\Q}{\mathbb{Q}}
    \newcommand{\Lra}{\Leftrightarrow}
    \newcommand{\al}{\alpha}
    \newcommand{\be}{\beta}
    \newcommand{\ga}{\gamma}
    \newcommand{\om}{\omega}
    \newcommand{\De}{\Delta}
    \newcommand{\oraw}{\overrightarrow}
    \newcommand{\bs}{\backslash}
    \newcommand{\2}{I\hspace{-1pt}I}
    \newcommand{\3}{I\hspace{-1pt}I\hspace{-1pt}I}

    \newtheorem{cs}{Case}
    \newtheorem{cas}{Case}
    \newtheorem{case}{Case}
    \newtheorem{apf}{別解}
    \newtheorem{anpf}{別解}

    \usetikzlibrary{calc}
    \title{2022年度京大数学(理系)の解答}
    \author{tt0801}
    \date{\today}
    
    \begin{document}
    \maketitle
    \section{大問1}
    \subsection{問題}
    $5.4 < \log_4 2022 < 5.5$であることを示せ. ただし, $0.301 < \log_{10} 2 < 0.3011$であることは用いてよい. 


    \subsection{解答}
    \begin{align*}
        \log_4 2022 &= \dfrac{\log_{10} 2022}{\log_{10} 4} \\
            &= \dfrac{\log_{10} 2 + \log_{10} 1011}{2\log_{10} 2} \\
            &= \dfrac{1}{2} + \dfrac{\log_{10} 1011}{2\log_{10} 2} \ (=A \mathrm{とおく})
    \end{align*}
    である. ここで, $\log_{10} 1011 > \log_{10} 1000 = 3$より, 
    \begin{align*}
        A & > \dfrac{1}{2} + \dfrac{3}{2\log_{10} 2} \\
        & > \dfrac{1}{2} + \dfrac{3}{2\cdot 0.3011} \ (\because \log_{10} 2 < 0.3011)\\
        &= \dfrac{33011}{6022} \\
        &> 5.4
    \end{align*}
    なので, $\log_4 2022 > 5.4$である. 
    
    また, $\log_{10} 1011 < \log_{10} 1024 = 10 \log_{10} 2$なので, 
    \begin{align*}
        A & < \dfrac{1}{2} + \dfrac{10 \log_{10} 2}{2\log_{10} 2} \\
        &= \dfrac{1}{2} + 5\\
        &= 5.5
    \end{align*}
    である. よって, $\log_4 2022 < 5.5$となる. 

    以上より, $5.4 < \log_4 2022 < 5.5$が示された. 


    \subsection{解説}
    $\log_{10} 2$の評価式が与えられているので, $\log_{10} 2$を作ることを意識して式変形をする. 

    \section{大問2}
    \subsection{問題}


    \subsection{解答}
    

    \subsection{解説}


    \section{大問3}
    \subsection{問題}
    $n$を自然数とする. 3つの整数$n^2+2$, $n^4+2$, $n^6+2$の最大公約数$A_n$を求めよ. 

    \subsection{解答}
    $(n^4+2) - (n^2+2) = n^2(n^2-1)$なので, $n^4+2$と$n^2+2$の最大公約数は, 
    $n^2+2$と$n^2(n^2-1)$の最大公約数に一致する. 

    $n^2+2$と$n^2$の最大公約数について考える. 
    $n^2+2$と$n^2$の公約数は, $(n^2+2)-n^2=2$を割り切る. 
    ゆえに, $n$が偶数の時は, $n^2+2$, $n^2$はともに偶数なので, 最大公約数は2である. 
    $n$が奇数の時は, $n^2+2$, $n^2$はともに奇数なので, 最大公約数は1である. 

    $n^2+2$と$n^2-1$の最大公約数について考える. 
    $n^2+2$と$n^2-1$の公約数は, $(n^2+2)-(n^2-1)=3$を割り切る. 
    ゆえに, $n$が3の倍数の時は, $n^2+2$, $n^2-1$はともに3の倍数ではないので, 最大公約数は1である. 
    $n$が3の倍数でない時は, $n^2+2$, $n^2-1$はともに3の倍数なので, 最大公約数は3である. 

    $n^2$と$n^2-1$は互いに素なので, $n^2+2$と$n^2(n^2-1)$の最大公約数は, 
    $n^2+2$と$n^2$の最大公約数と$n^2+2$と$n^2-1$の最大公約数の積である. 

    以上より, $n^2+2$と$n^2(n^2-1)$の最大公約数$B_n$は, 
    \begin{eqnarray*}
        B_n = \left\{
            \begin{array}{lll}
                2 & (n\mathrm{は偶数かつ3の倍数, すなわち6の倍数}), \\
                6 & (n\mathrm{は偶数だが3の倍数ではない, すなわち6で割って2または4余る}), \\
                1 & (n\mathrm{は奇数かつ3の倍数, すなわち6で割って3余る}), \\
                3 & (n\mathrm{は奇数だが3の倍数ではない, すなわち6で割って1または5余る})
            \end{array}
        \right.
    \end{eqnarray*}
    で与えられる. これが$n^4+2$と$n^2+2$の最大公約数に一致するので, 
    $A_n$は, $B_n$と$n^6+2$の最大公約数である. 
    
    一方で, $n^6+2 = n^2+2 + (n^2+1) \cdot n^2(n^2-1)$と表せるので, 
    $n^6+2$は, $n^2+2$と$n^2(n^2-1)$の最大公約数$B_n$で割り切れる. 

    以上より, $A_n = B_n$であるので, 
    \begin{eqnarray*}
        A_n = \left\{
            \begin{array}{lll}
                2 & (n\mathrm{は6の倍数}), \\
                6 & (n\mathrm{は6で割って2または4余る}), \\
                1 & (n\mathrm{は6で割って3余る}), \\
                3 & (n\mathrm{は6で割って1または5余る})
            \end{array}
        \right.
    \end{eqnarray*}
    である. 


    \subsection{解説}
    ユークリッドの互除法が背景にある問題である. 整数$x,y,q,r$について, $x = qy +r$
    と表せる時, $x,y$の最大公約数は$y,r$の最大公約数と一致する. 


    \section{大問4}
    \subsection{問題}
    四面体OABCが
    \begin{equation*}
        \mathrm{
            OA = 4,\ OB=AB=BC=3,\ OC=AC=2\sqrt{3}
        }
    \end{equation*}
    を満たしているとする. 点Pを辺BC上の点とし, $\triangle \mathrm{OAP}$の
    重心をGとする. このとき, 次の各問に答えよ. 
    \begin{itemize}
        \item [(1)] $\oraw{\mathrm{PG}} \perp \oraw{\mathrm{OA}}$を示せ. 
        \item [(2)] Pが辺BC上を動くとき, PGの最小値を求めよ. 
    \end{itemize}

    \subsection{解答}
    \begin{itemize}
        \item [(1)] 一次独立なベクトル$\oraw{\mathrm{OA}}$, $\oraw{\mathrm{OB}}$, $\oraw{\mathrm{OC}}$
        に着目する. 
        \begin{align*}
            \oraw{\mathrm{OA}} \cdot \oraw{\mathrm{OB}}
            &= \dfrac{|\oraw{\mathrm{OA}}|^2 + |\oraw{\mathrm{OB}}|^2 - |\oraw{\mathrm{AB}}|^2}{2} \\
            &= \dfrac{16 + 9 - 9}{2} \\
            &= 8, \\
            \oraw{\mathrm{OB}} \cdot \oraw{\mathrm{OC}}
            &= \dfrac{|\oraw{\mathrm{OB}}|^2 + |\oraw{\mathrm{OC}}|^2 - |\oraw{\mathrm{BC}}|^2}{2} \\
            &= \dfrac{9 + 12 - 9}{2} \\
            &= 6, \\
            \oraw{\mathrm{OA}} \cdot \oraw{\mathrm{OC}}
            &= \dfrac{|\oraw{\mathrm{OA}}|^2 + |\oraw{\mathrm{OC}}|^2 - |\oraw{\mathrm{AC}}|^2}{2} \\
            &= \dfrac{16 + 12 - 12}{2} \\
            &= 8.
        \end{align*}
        点Pは辺BC上にあるので, 0以上1以下の実数$t$を用いて
        \begin{equation*}
            \oraw{\mathrm{OP}} = t\oraw{\mathrm{OB}} + (1-t)\oraw{\mathrm{OC}}
        \end{equation*}
        と表せる. 点Gは$\triangle \mathrm{OAP}$の重心なので, 
        \begin{align*}
            \oraw{\mathrm{OG}} &= \dfrac{1}{3}\oraw{\mathrm{OA}} + \dfrac{1}{3}\oraw{\mathrm{OP}}
        \end{align*}
        だから, 
        \begin{align*}
            \oraw{\mathrm{PG}} &= \oraw{\mathrm{OG}} - \oraw{\mathrm{OP}} \\
            &= \dfrac{1}{3}\oraw{\mathrm{OA}} - \dfrac{2}{3}\oraw{\mathrm{OP}} \\
            &= \dfrac{1}{3}\oraw{\mathrm{OA}} - \dfrac{2}{3}t\oraw{\mathrm{OB}} - \dfrac{2}{3} (1-t)\oraw{\mathrm{OC}}
        \end{align*}
        である. 特に, $\oraw{\mathrm{OA}}$, $\oraw{\mathrm{OB}}$, $\oraw{\mathrm{OC}}$の係数が全て
        0になることはないので, 2点P, Gは一致しない. 
        \begin{align*}
            \oraw{\mathrm{PG}} \cdot \oraw{\mathrm{OA}} &=
            \dfrac{1}{3}|\oraw{\mathrm{OA}}|^2 
            - \dfrac{2}{3}t\oraw{\mathrm{OB}} \cdot \oraw{\mathrm{OA}} 
            - \dfrac{2}{3} (1-t)\oraw{\mathrm{OC}} \cdot \oraw{\mathrm{OA}} \\
            &= \dfrac{1}{3} \cdot 16 - \dfrac{2}{3}t \cdot 8 - \dfrac{2}{3} (1-t) \cdot 8 \\
            &= 0
        \end{align*}
        で, $\mathrm{PG} >0$, $\mathrm{OA}=4>0$なので, 
        $\oraw{\mathrm{PG}} \perp \oraw{\mathrm{OA}}$が導けた. 

        \item [(2)] $|\oraw{\mathrm{PG}}|^2$の最小値について考える. 
        \begin{align*}
            |\oraw{\mathrm{PG}}|^2 &= \oraw{\mathrm{PG}} \cdot \oraw{\mathrm{PG}} \\
            &= \oraw{\mathrm{PG}} \cdot \left(\dfrac{1}{3}\oraw{\mathrm{OA}} - \dfrac{2}{3}t\oraw{\mathrm{OB}} - \dfrac{2}{3} (1-t)\oraw{\mathrm{OC}} \right) \\
            &= - \dfrac{2}{3}t \oraw{\mathrm{PG}} \cdot \oraw{\mathrm{OB}} - \dfrac{2}{3} (1-t) \oraw{\mathrm{PG}} \cdot \oraw{\mathrm{OC}}
        \end{align*}
        である. 
        \begin{align*}
            \oraw{\mathrm{PG}} \cdot \oraw{\mathrm{OB}} &=
            \dfrac{1}{3}\oraw{\mathrm{OA}} \cdot \oraw{\mathrm{OB}} 
            - \dfrac{2}{3}t |\oraw{\mathrm{OB}}|^2
            - \dfrac{2}{3} (1-t)\oraw{\mathrm{OC}} \cdot \oraw{\mathrm{OB}} \\
            &= \dfrac{1}{3} \cdot 8 - \dfrac{2}{3}t \cdot 9 - \dfrac{2}{3} (1-t) \cdot 6 \\
            &= -2t - \dfrac{4}{3}, \\
            \oraw{\mathrm{PG}} \cdot \oraw{\mathrm{OC}} &=
            \dfrac{1}{3}\oraw{\mathrm{OA}} \cdot \oraw{\mathrm{OC}} 
            - \dfrac{2}{3}t \oraw{\mathrm{OB}} \cdot \oraw{\mathrm{OC}} 
            - \dfrac{2}{3} (1-t) |\oraw{\mathrm{OC}}|^2 \\
            &= \dfrac{1}{3} \cdot 8 - \dfrac{2}{3}t \cdot 6 - \dfrac{2}{3} (1-t) \cdot 12 \\
            &= 4t - \dfrac{16}{3} \\
        \end{align*}
        なので, 
        \begin{align*}
            |\oraw{\mathrm{PG}}|^2 
            &= - \dfrac{2}{3}t \oraw{\mathrm{PG}} \cdot \oraw{\mathrm{OB}} - \dfrac{2}{3} (1-t) \oraw{\mathrm{PG}} \cdot \oraw{\mathrm{OC}} \\
            &= - \dfrac{2}{3}t \left(-2t - \dfrac{4}{3}\right) - \dfrac{2}{3} (1-t) \left(4t - \dfrac{16}{3}\right) \\
            &= 4t^2 - \dfrac{16}{3}t + \dfrac{32}{9} \\
            &= 4 \left(t- \dfrac{2}{3}\right)^2 + \dfrac{16}{9} \\
            &\geq \dfrac{16}{9}. 
        \end{align*}
        等号成立条件は, $t=\dfrac{2}{3}$であり, これは$0\leq t\leq 1$を満たす. 
        このとき, 点Pは辺BCを1:2に内分する点である. また, $\mathrm{PG} = \sqrt{|\oraw{\mathrm{PG}}|^2 } = \dfrac{4}{3}$. 

        以上より, 点Pは辺BCを1:2に内分する点である時, PGは最小値$\dfrac{4}{3}$をとる. 

    \end{itemize}
    
    \subsection{解説}
    ベクトルの問題では, 基底(一次独立かつ, 空間の任意のベクトルが基底のベクトルの一次結合で表せる)
    に着目して式変形しよう. 本問では例えば, $\left\{\oraw{\mathrm{OA}}, \oraw{\mathrm{OB}}, \oraw{\mathrm{OC}}\right\}$
    が基底である. 
    
    \section{大問5}
    \subsection{問題}

    \subsection{解答}

    \subsection{解説}

    \section{大問6}
    \subsection{問題}

    \subsection{解答}

    \subsection{解説}

\end{document}